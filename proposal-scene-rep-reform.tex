\documentclass[10pt, a4paper]{article}

\usepackage{graphicx}
\usepackage{fullpage}
\usepackage{url}
\usepackage{parskip}

\title{Proposal: Restructure of the Committee of SVGE}
\author{Frederick Fillingham}

\begin{document}

\maketitle

\section{The Unique Case of SVGE:}
With such a wide variety of games with large secondary communities, many of which currently require a scene representative at the Committee level, the total count of Officer of the Committee roles is 19. Organising meetings of the Committee with such a large number of members is hard, trying to find a time where 19 people all have the same two or more hours available is nigh on impossible and always requires some members to be excluded in order to hold meetings. 

Six of the 19 Officers of the Committee are currently ``Scene Representatives'' and often do not feel the need to weigh in on discussions at meetings of the Committee. This document will lay out a new structure for the ``Scene Representative'' roles such that they are still communicating directly with the Committee Executive Branch but are not obliged to attend meetings of the Committee.


\section{Proposal:}

\subsection{Proposal Delayed Action: \label{sec:pda}}
This proposal, if voted into legislation in full or in parts is not to be written into the constitution until the next ``Annual General Meeting'', this is to avoid high levels of turbulence during times where normal operation of the committee is expected. Scene Representatives need not worry about their positions as Officers of the Committee.

\subsection{Removal of all ``Scene Representative'' \& ``Competition Officer'' roles from the main Committee: \label{sec:rem-scene-rep}}
To reduce the count of the Committee of SVGE, it is necessary to remove the positions of all Scene Representatives and Competition Officers from the main Committee Branch, this would lose them the opportunity to vote at meetings of the Committee. They will however maintain a number of abilities previously unique to Officers of the Committee, described in Section \ref{sec:new-com}.

\subsection{Addition of the ``Sub-Committee Representative'' role into the main Committee: \label{sec:new-rep}}
In order to properly represent the ``Scene Representatives'' and ``Competition Officers'' that are no longer on the main Committee, along with the interests of their respective communities, a new role will be created and will be elected from the pool of Officers of the ``Scene Representative Sub-Committee'' (Section \ref{sec:new-com}). The ``Sub-Committee Representative'' will have a vote at all meetings of the main Committee and will chair all subsequent meetings of the ``Scene Representative Sub-Committee''. They will also be expected to represent the interests of the ``Scene Representative Sub-Committee'' at all meetings of the main Committee.

\subsection{Formation of the ``Scene Representative Sub-Committee'': \label{sec:new-com}}
All Officer of the Committee roles that previously fell under the groups to be removed from the main Committee will be added to the ``Scene Representative Sub-Committee'', a group that operates with a reasonable degree of autonomy from the main Committee. The ``Scene Representative Sub-Committee'' will elect one of their own to take the new main Committee Branch role of ``Sub-Committee Representative'' (Section \ref{sec:new-rep}), this individual will be responsible for representing the ``Scene Representative Sub-Committee'' at meetings of the Committee Executive Branch.

Officers of the ``Scene Representative Sub-Committee'' will be permitted to attend and speak at meetings of the Committee in order to properly represent the ``Members of the Group'', though their attendance will not be required while scheduling a meeting of the committee. As mentioned, of the Officers within the ``Scene Representative Sub-Committee'', only the ``Sub-Committee Representative'' will have a vote during meetings of the Committee.

The process of election for the ``Sub-Committee Representative'' position is as follows and is to occur at the first meeting of the ``Scene Representative Sub-Committee'' where all Officers are able to attend. Officers of the ``Scene Representative Sub-Committee'' are required to select three candidates based on preference for the position. These are: first choice, second choice and absolute last choice. Candidates are then assigned a score based on the position they were placed in each Officer's vote, as follows: first placed gains ten Points, second placed gets five points and absolute last choice gains negative five points. Candidates may vote for themselves. Officers may not abstain from voting.

The candidate at the close of voting with the most points is then to be declared as the winner of the election and will be assigned the position of ``Sub-Committee Representative'' on top of their role as a ``Scene Representative''.


\section{Amending To This Proposal:\label{sec:amend}}
Any full member may propose amendments to this document and have them voted on under condition of implementation by Section \ref{sec:pda} as described in Section \ref{par:amending}.3.

\section{Voting On This Proposal:\label{sec:the-vote}}
All votes made over this proposal will be of standard form, as defined by Section 6.4 of the Constitution of SVGE. Votes will be taken on parts of this document in the order defined in Section \ref{sec:order-order}.

\subsection{The Order of Votes:\label{sec:order-order}}

\paragraph{\ref{sec:order-order}.1} First a vote on whether or not to enact this document in full is to be made, if this vote passes then subsequent votes are not required and proceedings may move to Section \ref{par:amending}. In the case of this vote failing, the votes shall follow all those described below.

\paragraph{\ref{sec:order-order}.2} Votes will now be held on the conditions that if any part of this document passes a vote, Section \ref{sec:pda} will be passed along with it to ensure turbulent conditions for ``Scene Representatives'' and the committee in general are avoided. Voting will now begin for each section of the document from \ref{sec:rem-scene-rep} to \ref{sec:new-com}. If this proposal does not pass voting in a form that would be deemed operable, it is to be abandoned in completeness, this condition will be defined by the meeting chair.

\paragraph{\ref{sec:order-order}.3\label{par:amending}} Votes will now be held over each of the proposals of amendment made for this document, these votes will be held in the order of final submission of amendments.






\end{document}